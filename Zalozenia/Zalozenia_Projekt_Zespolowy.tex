% !TeX encoding = UTF-8
% !TeX spellcheck = pl_PL

%% Konfiguracja:
\newcommand{\kurs}{Projekt Zespo\l{}owy}
\newcommand{\formakursu}{Projekt}

%etap skladania dokumentu
\newcommand{\doctype}{Za\l{}o\.{z}enia projektowe}

%wpisz nazwę projektu
\newcommand{\projectname}{Urządzenie badające mikroklimat pomieszczenia biurowego w celu zwiększenia komfortu i efektywności pracy}

%wpisz akronim projektu
%\newcommand{\acronim}{ZuO}

%wpisz Imię i nazwisko oraz numer albumu
\newcommand{\osobaC}{Albert \textsc{Lis}, 235534}
\newcommand{\osobaE}{Michał \textsc{Moruń}, 235986}
\newcommand{\osobaB}{Paula \textsc{Langkafel}, 235373}
\newcommand{\osobaA}{Krzysztof \textsc{Kurnik}, 237603}
\newcommand{\osobaD}{Maciej \textsc{Maruszak}, 235437}

%wpisz termin w formie, jak poniżej dzień, parzystość, godzina
\newcommand{\termin}{śr 8:15}

%wpisz imię i nazwisko prowadzącego
\newcommand{\prowadzacy}{dr in\.{z}. Krzysztof \textsc{Arent}}

\documentclass[10pt, a4paper]{article}
\usepackage{pgfgantt}
\usepackage{blindtext}
\usepackage{enumitem}
\usepackage{xcolor}
\include{preambula_projekt_zespolowy}
	
\begin{document}

\def\tablename{Tabela}	%zmienienie nazwy tabel z Tablica na Tabela

\begin{titlepage}
	\begin{center}
		\textsc{\LARGE \formakursu}\\[1cm]		
		\textsc{\Large \kurs}\\[0.5cm]		
		\rule{\textwidth}{0.08cm}\\[0.4cm]
		{\huge \bfseries \doctype}\\[1cm]
		{\huge \bfseries \projectname}\\[0.5cm]
%		{\huge \bfseries \acronim}\\[0.4cm]
		\rule{\textwidth}{0.08cm}\\[1cm]
		
		\begin{flushright} \large
		\emph{Skład grupy:}\\
		\osobaA\\
		\osobaB\\
		\osobaC\\
		\osobaD\\
		\osobaE\\[0.4cm]
		
		\emph{Termin: }\termin\\[0.4cm]

		\emph{Prowadzący:} \\
		\prowadzacy \\
		
		\end{flushright}
		
		\vfill
		
		{\large \today}
	\end{center}	
\end{titlepage}

\newpage
\tableofcontents
\newpage

\section{Opis projektu}
\label{sec:OpisProjektu}

Urządzenie badające mikroklimat w pomieszczeniu biurowym w celu zwiększenia komfortu i efektywności pracy.

Przedmiotem projektu jest sprawdzanie i optymalizacja warunków panujących w pomieszczeniach biurowych, na podstawie badań z szerokiego zakresu nauki jaką jest ergonomia, a także norm prawnych. 
Praca w odpowiednich warunkach pozwala na skuteczniejsze zarządzanie swoim czasem oraz efektywniejsze
wykonywanie postawionych zadań. Często jednak zarówno pracownicy jak i pracodawcy nie zdają sobie sprawy jak niewłaściwe warunki środowiskowe wpływają na jakość pracy. Objawami niekorzystnych parametów panujących w biurze może być brak energii, senność i uczucie zmęczenia. Nawet jeżeli pracownik wykryje zbyt wysoką temperaturę lub niewystarczające oświetlenie bywa, że ignoruje te sygnały pochłonięty pracą. Przedstawiane urządzenie go wyręczy i rozszerzy zakres detekcji podstawowych czynników informując o niewłaściwych parametrach.

Wykorzystane zostaną następujące czujniki:

\begin{description}[font=$\bullet$~\normalfont]
\item temperatury
\item wilgotności
\item ciśnienia
\item dwutlenku/tlenku węgla
\item natężenia światła
\item hałasu
\end{description}

Urządzenie będzie w formie małego pudełka, które będzie swobodnie spoczywać na biurku lub innej powierzchni w badanym pomieszczeniu. Jedynym wymogiem co do ułożenia będzie czujnik oświetlenia, który powinien być skierowany w górę. Zasilanie całości będzie zrealizowane bezpośrednio z sieci 230V, poprzez zasilacz wyposarzony w przetwornicę prądowo napięciową, dostaraczającą do samego urządzenia około 5V.

Zbieranie danych będzie odbywać się w sposób ciągły. Pomiary, poprzez wbudowany moduł WiFi, będą bezpośrednio przesyłane na serwer. Po dokonaniu analizy użytkownik zostanie poinformowany o jakości klimatu w pomieszczeniu i ewentualnej potrzebie reagowania. Informacje będą udostępnione na stronie internetowej serwera, gdzie zostanie stworzony intuicyjny interfejs użytkownika, umożliwający bezpośrednią analizę pozyskanych aktualnych pomiarów jak i wcześniejszych.

\section{Kryteria sukcesu}
Celem projektu będzie badanie warunków panujących w pomieszczeniach. Testem sprawdzającym poprawne działania wytworzonego produktu będzie wniesienie urządzenia do pomieszczenia, które zostało świeżo wywietrzone. Zostaną zamknięte drzwi oraz okna na czas około 20-30 minut. W pokoju będzie przebywać kilka osób. Pomiary będą dokonywane na bierząco. Kryterium sukcesu zostanie osiągnięte, po spełnieniu następujących warunków:

\begin{description}[font=$\bullet$~\normalfont]
\item Urządzenie pomiarowe wykryje zmiany zachodzące w pomieszczeniu.
\item Urządzenie we właściwy sposób wskaże co należy zrobić w celu poprawy warunków w pomieszczeniu.
\end{description}

\section{Plan pracy i rozklad w czasie}

Początkowo projekt zostanie podzielony na 5 etapów, wyróżniając 4 kamienie milowe oraz końcowe oddanie urządzenia. 
\subsection{Kamienie milowe}

\subsubsection{Wymagania użytkownika i założenia}
Pierwszy etap będzie skupiał się na pojęciu ergonomii czyli nauki o pracy. Zostaną zebrane informacje, powołujące sie na badania i przepisy prawne dotyczące właściwych warunków pracy. 
Po określeniu zapotrzebowań użytkowników, zostaną wybrane czynniki środowiskowe które będą mierzone. Badane poszczególne wartości posłużą do analizy zbieranych danych, a następnie do informowania użytkownika o ewentualnych możliwościach polepszenia aktualnie panujących warunków.

\begin{description}[font=$\bullet$~\normalfont]
\item Kryterium sukcesu:
	\begin{description}[font=$\bullet$~\normalfont]
	\item Wybranie znaczączych współczynników środowiskowych.
	\item Dobranie przedziałów poszczególnych parametrów.
	\item Ustalenie działań potrzebnych do polepszenia warunków w pomieszczeniu.
	\end{description}
\end{description}


\subsubsection{Projekt, implementacja komponentów i systemu}
Wraz z drugim etapem rozpocznie się techniczna część realizaji projektu. Zbudowanie samego urządzenia, wstępnie zbierającego dane na dysk lokalny, a następnie dodanie funkcjonalności w postaci wysyłania danych na serwer. Bazową płytką, do której będą podłączonego wszystkie czujniki, będzie Arduino Mega, umożliwiające podłączenie oraz oprogramowanie wszystkich podzespołów. Analiza danych poimarowych będzie odbywać się po stronie urządzenia, wyświetlając komunikaty dla użytkownika na wyświetlaczu OLED. Pomimo tego dane będą wysyłane na serwer.
Następnie zostanie oprogramowana obsługa danych z czujników w przestrzeni serwerowej. Wiąże się to z utworzeniem bazy danych wszystkich pomiarów z wyszczególnionym dniem i godziną. Następnie zostanie stworzony interfejs dla użytkowników oraz administratorów systemu. W aplikacji webowej będą dostępne informacje nt. warunków panujących w pomieszczeniu oraz sugestie dla użytkowników.

\begin{description}[font=$\bullet$~\normalfont]
\item Kryterium sukcesu:
	\begin{description}[font=$\bullet$~\normalfont]
	\item Poprawne zbieranie danych przez czujniki.
	\item Odpowiednie wyświetlanie sugestii na wyświetlaczu OLED.
	\item Właściwe odbieranie danych przez serwer i zapis do bazy danych.
	\item Działająca wstępna wersja aplikacji webowej.
	\item Projekt obudowy i wyglądu urządzenia.
	\end{description}
\end{description}


\subsubsection{Integracja systemu}
Ten etap posłuży do gruntownego sprawdzenia i przebadania wszystkich funkcjonalności urządzenia i systemu. Będzie to ostatni etap wprowadzania zmian i poprawek w już działającym urządzeniu. Ewentualnie projekt zostanie rozszerzony o dodatkowe funkcjonalności, które wynikną w trakcie testowania urządzenia.

\begin{description}[font=$\bullet$~\normalfont]
\item Kryterium sukcesu:
	\begin{description}[font=$\bullet$~\normalfont]
	\item Zapewnienie pełnej funkcjonalności urządzenia.
	\item Możliwość przeglądania danych w aplikacji webowej.
	\item Pomyślnie przeprowadzony test działania.
	\end{description}
\end{description}


\subsubsection{Ewaluacja}
Etap będzie składał się głównie z podsumowania całego projektu. Zrobienia końcowej dokumentacji, wyciągnięcia wniosków z przebiegu wszystkich prac.
\begin{description}[font=$\bullet$~\normalfont]
\item Kryterium sukcesu:
	\begin{description}[font=$\bullet$~\normalfont]
	\item Pomyślne zakończenie całego projektu.
	\end{description}
\end{description}

\subsection{Dekompozycja projektu}

\subsubsection{Wymagania użytkownika i założenia}
\begin{enumerate}
\item Zebranie badań dotyczących ergonomii
\item Wyszczególnienie najważniejszych współczynników
\item Określenie zakresu współczynników
\item Wybranie odpowiedniej formy zachowania się w przypadku przekroczenia dopuszczalnej wartości
\end{enumerate}
\subsubsection{Projekt, implementacja komponentów i systemu}
\begin{enumerate}
\item Podłączenie i oprogramowanie czujnika:
	\begin{description}[font=$\bullet$~\normalfont]
	\item temperatury i wilgotności
	\item ciśnienia
	\item dwutlenku/tlenku węgla
	\item natężenia światła
	\item hałasu
	\end{description}
\item Stworzenie logiki w oparciu o ustalone zakresy
\item Dodanie sygnalizacji na ekranie OLED
\item Podłączenie i oprogramowanie modułu WiFi ESP8266
\item Back-end
	\begin{enumerate}
	\item Odbieranie danych z Arduino
	\item Zapisywanie odebranych pomiarów
	\item Utworzenie bazy danych
	\item Klasyfikacja pomiarów w bazie danych
	\end{enumerate}
\item Front-end	
	\begin{enumerate}
	\item Wyświetlanie danych w formie tabeli na stronie
	\item Stworzenie interfejsu graficznego pokazującego pomiary w formie wykresów
	\item Stworzenie systemu powiadomień w wypadku przekroczenia dopuszczalnych wartości
	\end{enumerate}
\end{enumerate}

\subsection{Wykres Gantta}


\hspace{-60px}
\begin{ganttchart}[hgrid,vgrid]{1}{33}
\gantttitle{2019}{33}{14} \\
\gantttitlelist{1,...,11}{3} \\
\ganttgroup{3.1.1}{1}{4} \\
\ganttgroup{3.1.2}{6}{22} \\
\ganttgroup{3.1.3}{24}{28} \\
\ganttgroup{3.1.4}{30}{32} 
\end{ganttchart}


\section{Doręczenie}

\subsection{Wymagania użytkownika i założenia}
	\begin{enumerate}
	\item Opis zmiennych środowiskowych wpływających na samopoczucie i zdrowie w miejscu pracy, w tym określenie skutków niezapewnienia odpowiednich warunków.
	\item Wyszczególnione odpowiednie zakresy zmiennych środowiskowych.
	\item Opis sposobów reagowania w celu poprawienia warunków panujących w pomieszczeniu
	\end{enumerate}
\subsection{Projekt, implementacja komponentów i systemu}
	\begin{enumerate}
	\item Opis i parametry układu sterującego oraz czujników
	\item Opis sposobu komunikacji z serwerem
	\item Archiwum z programem obsługującym urządzenie
	\item Struktura bazy danych czujników
	\item Aplikacja webowa z interfejsem użytkownika
	\item Archiwum z oprogramowaniem aplikacji webowej i bazy danych
	\end{enumerate}
\subsection{Integracja systemu}
\begin{enumerate}
	\item Raport z przeprowadzonych badań
	\end{enumerate}
\subsection{Ewaluacja}
	\begin{enumerate}
	\item Całościowa dokumentacja projektu
	\end{enumerate}

\section{Budżet}
\subsection{Elektronika - czujniki}

\begin{table}[H]
		\centering
		\begin{tabular}{|r|l|l|} \hline
			\textbf{Wielkość mierzona} & \textbf{Symbol czujnika} & \textbf{Cena}\\
			\hline
			Wilgotność + Temperatura & SHT30 & 16,50zł \\
			Ciśnienie & Bmp180 & 32,90zł\\
			Smog & GP2Y1010AU0F & 50,90zł \\
			Dym & MQ-9 & 27,99zł \\
			Tlenek węgla (CO) & MQ-7 & 13,99zł\\
			Dwutlenek węgla (CO$_2$) & MH-Z19 & 150,56zł\\
			Natężenie światła & BH1750FVI & 15,99zł\\ 
			\hline
		\end{tabular}
		\caption{Użyte czujniki}
		\label{tab:Czujniki}
	\end{table}


\subsection{Elektronika - sterownie i komunikacja}

\begin{table}[H]
		\centering
		\begin{tabular}{|r|l|l|} \hline
			\textbf{Urządzenie} & \textbf{Symbol} & \textbf{Cena}\\
			\hline
			Arduino Mega & Arduino MEGA 2560 R3 & 39,99zł \\
			Moduł sieciowy WiFi & ESP8266 & 16,49zł\\
			Wyświetlacz & OLED 0.96 SPI biały & 39,90zł \\ \hline
		\end{tabular}
		\caption{Moduły sterownia i komunikacji}
		\label{tab:Sterowanie}
	\end{table}
	
\subsection{Zasoby ludzkie}

\begin{table}[H]
		\centering
		\begin{tabular}{|r|l|} \hline
			\textbf{Stanowisko} & \textbf{Roboczo godzina (Brutto)} \\
			\hline
			Programista C/C++ & 31,07zł\\
			Elektronik & 25,06zł\\
			Specjalista ds. BHP& 19,70zł\\ 
			Administrator baz danych & 23,81zł\\
			Administrator Serwera & 29,76zł\\ 
			\hline
		\end{tabular}
		\caption{Zasoby ludzkie}
		\label{tab:Ludzie}
	\end{table}

\section{Zarządzanie projektem}
\begin{description}[font=$\bullet$~\normalfont]
\item Raporty, oddawane podczas oddawania kamieni milowych na spotkaniach, będą nadzorowane poprzez poszczególne osoby odpowiedzialne za dany etap projektu wraz z koordynatorem zespołu.
\item Zespół projektowy zakłada pracę według metodyki Scrum, dekomponując poszczególne kamienie milowe na mniejsze zadania, które będą wykonywane podczas tzw. sprintów trwających od tygodnia do czterech.
\item Spotkania zespołu będą odbywać się co tydzień lub dwa, data i miejsce kolejnego zebrania członków zespołu projektowego będzie umawiana podczas spotkań
\item Dokumentacja oraz aktualne postepy w pracach zespołu projektowego będą udostępniane przy użyciu następujących narzędzi: eportal.pwr.edu.pl, Slack, taiga.io.
\item Konflikty w zespole będą rozstrzygane przy pomocy mediacji w obecności wszystkich członków. \item Decyzje dotyczące podziału zadań oraz kolejnych przedsięwzięć będą podejmowane wspolnie przez wszystkich członków zespołu.
\item Prawa własności intelektualnej wytworzonego urządzenia, oraz oprogramowania podzielone zostaną pomiędzy wszystkich członków zespołu, który je wytworzył. Dalsza eksploatacja, rozwój  i udostępnianie osobom trzecim może się odbyć tylko i wyłącznie za zgodą pozostałych członków zespołu. W przypadku zakupu i użycia części sponsorowanych przez Politechnika Wrocławska, konkretne podzespoły zostaną przekazane prowadzącemu kurs "Projekt Zespołowy" wraz z końcem semestru.
\end{description}


\section{Zespół}
\begin{description}[font=$\bullet$~\normalfont]
\item Koordynator zespołu: Krzysztof Kurnik nr. indeksu: 237603 email: 237603@student.pwr.edu.pl
\end{description}
Koordynator zespołu projetowego jest opowiedzialny za organizację pracy zespołu oraz za kolejne etapu tworzenia produktu. Poniżej została podana lista członków zespołów wraz z przypsianiem odpowiedzialności za poszczególny etap projektu.
\begin{description}[font=$\bullet$~\normalfont]
\item Inżynier systemu: Paula Langkafel nr. indeksu: 235373 email: 235373@student.pwr.edu.pl
\item Developer oprogramowania: Albert Lis nr. indeksu: 235534 email: 235534@student.pwr.edu.pl
\item Developer/Konstruktor: Michał Moruń nr. indeksu: 235986 email: 235986@student.pwr.edu.pl
\item Inżynier systemu: Maciej Maruszak nr. indeksu: 235437 email: 235437@student.pwr.edu.pl
\end{description}

\end{document}
