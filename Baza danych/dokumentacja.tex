% !TeX encoding = UTF-8
% !TeX spellcheck = pl_PL

% $Id:$

%Author: Wojciech Domski
%Szablon do ząłożeń projektowych, raportu i dokumentacji z steorwników robotów
%Wersja v.1.0.0
%


%% Konfiguracja:
\newcommand{\kurs}{Projekt zespołowy}
\newcommand{\formakursu}{Projekt}

%odkomentuj właściwy typ projektu, a pozostałe zostaw zakomentowane
%\newcommand{\doctype}{Za\l{}o\.{z}enia projektowe} %etap I
%\newcommand{\doctype}{Raport} %etap II
\newcommand{\doctype}{Dokumentacja} %etap III

%wpisz nazwę projektu
\newcommand{\projectname}{Baza danych}

%wpisz akronim projektu
%\newcommand{\acronim}{Mikruś}

%wpisz Imię i nazwisko oraz numer albumu
\newcommand{\osobaA}{Albert \textsc{Lis}, 235534}
%w przypadku projektu jednoosobowego usuń zawartość nowej komendy
%\newcommand{\osobaB}{Bart\l{}omiej \textsc{Cabacki}, 123457}

%wpisz termin w formie, jak poniżej dzień, parzystość, godzina
%\newcommand{\termin}{Śr 13:05}

%wpisz imię i nazwisko prowadzącego
%\newcommand{\prowadzacy}{drr in\.{z}. Paweł \textsc{Drąg}}

\documentclass[10pt, a4paper]{article}

\include{preambula}
	
\begin{document}

\def\tablename{Tabela}	%zmienienie nazwy tabel z Tablica na Tabela

\begin{titlepage}
	\begin{center}
		\textsc{\LARGE \formakursu}\\[1cm]		
		\textsc{\Large \kurs}\\[0.5cm]		
		\rule{\textwidth}{0.08cm}\\[0.4cm]
		{\huge \bfseries \doctype}\\[1cm]
		{\huge \bfseries \projectname}\\[0.5cm]
	%	{\huge \bfseries \acronim}\\[0.4cm]
		\rule{\textwidth}{0.08cm}\\[1cm]
		
		\begin{flushright} \large
%		\emph{Skład grupy:}\\
		\osobaA\\[0.4cm]

		
%		\emph{Termin: }\termin\\[0.4cm]

%		\emph{Prowadzący:} \\
%		\prowadzacy \\
		
		\end{flushright}
		
		\vfill
		
		{\large \today}
	\end{center}	
\end{titlepage}

\newpage
%\tableofcontents

\section{Struktura bazy danych}
	Baza stworzona w języku MySQL. Posiada 4 tabele:
	\begin{enumerate}
		\item measurements
		\newline
		Tabela główna posiada informację na temat pomiarów oraz dat.
		
		\item rooms
		\newline
		Tabela przechowująca informację na temat pokojów w których zostały dokonane pomiary.
		
		\item sensors
		\newline
		Tabela przechowująca informację na temat czujników.
	\end{enumerate}

\begin{figure}[H]
	\centering
	\includegraphics[width=1\textwidth]{figures/diag1.png}
	\caption{Architektura bazy danych}
	\label{fig:ArchitekturaBD2}
\end{figure}

%Obecne w dokumencie do etapu II oraz III
\section{Komunikacja między sterownikiem a bazą danych}
Do tego zadania został wykorzystany program napisany w języku Python w oparciu o bibliotekę: mysql connector \cite{mysql} oraz port szeregowy do komunikacji między sterownikiem a PC. Docelowo komunikacja będzie prowadzona przez UDP i moduł sieciowy.

\begin{thebibliography}{9}
	
	\bibitem{mysql}
	\href{https://dev.mysql.com/doc/connector-python/en/}{Biblioteka MySQL dla języka Python}
	
\end{thebibliography}


\end{document}







































