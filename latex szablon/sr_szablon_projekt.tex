% !TeX encoding = UTF-8
% !TeX spellcheck = pl_PL

% $Id:$

%Author: Wojciech Domski
%Szablon do ząłożeń projektowych, raportu i dokumentacji z steorwników robotów
%Wersja v.1.0.0
%


%% Konfiguracja:
\newcommand{\kurs}{Projekt Zespo\l{}owy}
\newcommand{\formakursu}{Projekt}

%odkomentuj właściwy typ projektu, a pozostałe zostaw zakomentowane
\newcommand{\doctype}{Ergonomia} %etap I
%\newcommand{\doctype}{Raport} %etap II
%\newcommand{\doctype}{Dokumentacja} %etap III

%wpisz nazwę projektu
\newcommand{\projectname}{Smog i ciśnienie}

%wpisz akronim projektu
%\newcommand{\acronim}{ZuO}

%wpisz Imię i nazwisko oraz numer albumu
\newcommand{\osobaC}{Albert \textsc{Lis}, 235534}
%w przypadku projektu jednoosobowego usuń zawartość nowej komendy

%wpisz termin w formie, jak poniżej dzień, parzystość, godzina
\newcommand{\termin}{śr 8:15}

%wpisz imię i nazwisko prowadzącego
\newcommand{\prowadzacy}{mgr in\.{z}. Krzysztof \textsc{Arent}}

\documentclass[10pt, a4paper]{article}

\include{preambula}
	
\begin{document}

\def\tablename{Tabela}	%zmienienie nazwy tabel z Tablica na Tabela

\begin{titlepage}
	\begin{center}
		\textsc{\LARGE \formakursu}\\[1cm]		
		\textsc{\Large \kurs}\\[0.5cm]		
		\rule{\textwidth}{0.08cm}\\[0.4cm]
		{\huge \bfseries \doctype}\\[1cm]
		{\huge \bfseries \projectname}\\[0.5cm]
		%{\huge \bfseries \acronim}\\[0.4cm]
		\rule{\textwidth}{0.08cm}\\[1cm]
		
		\begin{flushright} \large
		%\emph{Skład grupy:}\\

		\osobaC\\

		
		\emph{Termin: }\termin\\[0.4cm]

		\emph{Prowadzący:} \\
		\prowadzacy \\
		
		\end{flushright}
		
		\vfill
		
		{\large \today}
	\end{center}	
\end{titlepage}

%\newpage
%\tableofcontents
%\newpage

%Obecne we wszystkich dokumentach
\section{Smog}
	\subsection{Opis}
	Najlepsze w sensie zarówno zdrowotnym jak i wydajnościowym dla człowieka jest zerowe zanieczyszczenie powietrza. Jednak jest to niemożliwe do osiągnięcia bez specjalnych urządzeń filtrujących. Dlatego stosuje się odpowiednie normy określające dozwolony poziom zanieczyszczeń. Rozróżnia się 2 kryteria zanieczyszczenie uśrednione na dobę oraz na rok. W przypadku pracy biurowej istotniejsze jest zanieczyszczenie dzienne gdyż wpływa ono bezpośrednio na samopoczucie pracowników. Istnieje wiele różnych norm regulujących poziom zanieczyszczeń. Zdecydowaliśmy się wybrać wyznaczone przez WHO\cite{WHO}. Nasza decyzja uwzględniła to że jest to obecnie najrestrykcyjniesza norma i prawdopodobnie odporna na wpływy polityczne. Np. norma polska jest zawyżona i w ten sposób można osiągnąć lepsze statystyki jako państwo. Obecnie monitoruje się frakcje o średnicy pyłu 2,5$\mu m$ i 10$\mu m$. Ale warto również zwrócić uwagę na 1$\mu m$ i 0.3$\mu$um gdyż te cząstki są najbardziej szkodliwe. Niestety ze względu na użyty czujnik nie jesteśmy w stanie zmierzyć zanieczyszczeń poniżej 1$\mu m$.
	
	\subsection{Możliwości poprawienia warunków}
	Zakupienie odpowiedniego urządzenia filtrującego powietrze i spełniającego odpowiednie normy filtracji. Sugerowany przedział nrom dla filtrów: F9 – H14. Obecnie takie rozwiązania nie posiadają wygórowanej ceny. Tańszą alternatywą (gdy budynek posiada wentylację) jest zamontowanie takiego filtru na drodze świeżego powietrza. Wtedy koszt takiego filtru wynosi $\sim$ 15zł/m$^2$.
	
\section{Ciśnienie}
	\subsection{Opis}
	Optymalne ciśnienie dla człowieka wynosi 1016hPa. W przytoczonym przeze mnie badaniu\cite{Badanie} sprawdzano wpływ ciśnienia na ilość występowania zawałów mięśnia sercowego. Badanie zawiera dane zbierane w ciągu 10 lat i przeanalizowano prawie 260000 pacjentów. Dla ciśnienia atmosferycznego występuje tzw. model V. Gdzie minimalna częstość zawałów występowała przy ciśnieniu 1016hPa. Wzrost ilości występuje zarówno w przypadku wyższego ciśnienia jak i niższego. Różnica 10hPa w obie strony powodowała wzrost częstości zdarzeń o ok 12$\%$.
	\subsection{Możliwości poprawienia warunków}
	Niestety możliwości regulacji ciśnienia są znacznie ograniczone. Negatywne efekty można próbować zniwelować odpowiednio sterując klimatyzacją/nawiewem. W przypadku niskiego ciśnienia można więcej powietrza wtłaczać do budynku niż wyciągać i odwrotnie.
	
\begin{thebibliography}{9}
	
	\bibitem{WHO}
	\href{https://www.who.int/news-room/fact-sheets/detail/ambient-(outdoor)-air-quality-and-health}{Ambient (outdoor) air quality and health}

	\bibitem{Badanie}
	\href{https://www.ahajournals.org/doi/full/10.1161/01.CIR.100.1.e1}{Unhealthy Effects of Atmospheric Temperature and Pressure on the Occurrence of Myocardial Infarction and Coronary Deaths.}
	
\end{thebibliography}

\end{document}







































